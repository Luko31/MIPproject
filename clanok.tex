\documentclass[10pt,twoside,slovak,a4paper]{article}

\usepackage[slovak]{babel}
%\usepackage[T1]{fontenc}
\usepackage[IL2]{fontenc} % lepšia sadzba písmena Ľ než v T1
\usepackage[utf8]{inputenc}
\usepackage{graphicx}
\usepackage{url} % príkaz \url na formátovanie URL
\usepackage{hyperref} % odkazy v texte budú aktívne (pri niektorých triedach dokumentov spôsobuje posun textu)

\usepackage{cite}
%\usepackage{times}

\pagestyle{headings}

\title{Získavanie a spracovanie dát obsadenosti spojov v regionálnej doprave s cieľom skvalitnenia služieb\thanks{Semestrálny projekt v predmete Metódy inžinierskej práce, ak. rok 2025/26, vedenie: Ing. Ivan Kapustík}}

\author{Lukáš Baran,
Jonáš Béreš,
Peter Bódi,
Martin Tacsik\\[2pt]
	{\small Slovenská technická univerzita v Bratislave}\\
	{\small Fakulta informatiky a informačných technológií}\\
	{\small \texttt{xtacsik@stuba.sk}}
	}

\date{\small 7. 10. 2025}



\begin{document}

\maketitle



\begin{abstract}
%akronym
ZDOS


%anotacia
Projekt sa zameriava na analýzu obsadenosti spojov v regionálnej doprave s cieľom navrhnúť spôsoby zlepšenia kvality služieb pre cestujúcich. 
Sucastou projektu je analyza ziskanych dat za ucelom skvalitnenia sluzieb verejnej dopravy . Zber dát bude realizovaný pomocou viacerych metod,  
ktore budu sucastne implementovane do vozidiel verejnej dopravy. Medzi metody skumane v tomto projekte su: zaznam z kamier, pohybove senzory, 
termalne senzory, prerusovacie senzory,   prepojenie s aplikacou a terminalom vo vozidlach verejnej dopravy . Projekt sa zameriava na zistenie 
optimalneho sposobu zberu relelvantych dat. Získané dáta budú pripravené na následovné spracovanie - skúmanie korelácie medzi predpokladanou a 
reálnou obsadenosťou spojov. Vyhodnocovanie dát v reálnom čase a ich grafické zobrazenie pre verejnost a prevadzkovatelov. Zamerom je aj porovnanie 
aktuálne dostupných metód s historicky využívanými metódami merania obsadenosti verejnej dopravy.

%\ldots
\end{abstract}
\end{document}


\section{Úvod}

Cieľom tohto projektu je prispieť k zlepšeniu kvality služieb v regionálnej doprave prostredníctvom efektívneho získavania a spracovaniadát o obsadenosti spojov. V súčasnosti väčšina 
dopravcov v regionálnej autobusovej či železničnej doprave nevyužíva detailné, v reálnom čase 
dostupné informácie o počte cestujúcich. Tieto údaje sú pritom kľúčové nielen pre optimalizáciu prevádzky a plánovania spojov, ale aj pre komfort 
cestujúcich, ktorí by vďaka nim mohli lepšie plánovať svoju cestu. Projekt sa preto zameriava na komplexný pohľad na problematiku – od identifikácie
dostupných technológií až po návrh spôsobu spracovania a prezentácie dát.

Uveďte explicitne štruktúru článku. Tu je nejaký príklad.
Základný problém, ktorý bol naznačený v úvode, je podrobnejšie vysvetlený v časti~\ref{nejaka}.
Dôležité súvislosti sú uvedené v častiach~\ref{dolezita} a~\ref{dolezitejsia}.
Záverečné poznámky prináša časť~\ref{zaver}.



\section{Naše ciele} \label{nejaka}

Z obr.~\ref{f:rozhod} je všetko jasné. 

\begin{figure*}[tbh]
\centering
%\includegraphics[scale=1.0]{diagram.pdf}
Hlavným cieľom tohto projektu je prispieť k zlepšeniu kvality služieb v regionálnej doprave prostredníctvom efektívneho získavania a spracovania dát
o obsadenosti spojov. Zatial čo sa nejedná o úplne nový koncept, myslíme si, že v súčasnosti veľa dopravcov  nevyužíva detailné, v reálnom čase dostupné informácie o počte cestujúcich.
Tieto údaje sú pritom kľúčové nielen pre optimalizáciu prevádzky a plánovania spojov, ale aj pre komfort 
cestujúcich, ktorí by vďaka nim mohli lepšie plánovať svoju cestu. Projekt sa preto zameriava na komplexný pohľad na problematiku – od identifikácie
dostupných technológií až po návrh spôsobu spracovania a prezentácie dát. Nižšie si detailnejšie rozoberieme jednotlivé ciele projektu. 

1. Zlepšenie kvality služieb pre cestujúcich

Prvým a hlavným cieľom je zvýšenie kvality služieb poskytovaných v regionálnej doprave. Získavaním údajov o obsadenosti je možné poskytnúť citelné zlepšenie kvality a komfortu 
na viacerých frontoch. Získané dáta by sa v reálnom čase mohli využiť na optimalizáciu plánovania takzvaných "posilových spojov" (nepravidelné spoje, ktoré na istú dobu zvyšujú kapacitu
danej linky) a poskytovanie presných informácii o aktuálnej dopravnej situácii cestujúcim.

➡️ Reálnosť dosiahnutia: Tento cieľ je realistický v dlhodobom horizonte, avšak konečná možnosť implementácie závisí od dostupnosti technológii a najmä ochoty
zo strany prevádzkovateľov investovať do skvalitňovania služieb a doby montáže nových zariadení do ich vozidiel. Odhadovaný horizont na splnenie cieľu by bol až niekoľko rokov.

2. Získanie informácií o aktuálnych metódach zisťovania obsadenosti

Ďalším cieľom je preskúmať existujúce a aktívne využívané prístupy k zisťovaniu obsadenosti spojov – manuálne aj automatizované systémy. Chceme zistiť, ktoré metódy sa využívajú v podobných 
oblastiach, ako aj dôvod ich implementácie a hlavne aké výhody a nevýhody poskytujú. 
➡️ Reálnosť dosiahnutia: Cieľ je splniteľný. Jedná sa o teoretickú časť, ktorá vyžaduje rešerš aktuálne používaných metód. 

3. Identifikácia technických možností získavania dát

Projekt sa zaoberá aj popisom analýzou a vyhodnocovaním vhodnosti metód na získavanie dát obsadenosti vozidiel verejnej dopravy. Môže ísť o infračervené senzory, 
váhové senzory v podlahe alebo na sedadlách, kamerové systémy, či dokonca sledovanie pripojených mobilných zariadení. Cieľom projektu je objaviť čo najvhodnejšie celoplošné 
riešenie na zber dát. Metódy ako manuálne rátanie cestujúcich alebo označovanie lístkov nepovažujeme za dostatočne efektívne pre dnešnú spoločnosť a preto sa im nebudeme v hĺbke venovať.
➡️ Reálnosť dosiahnutia: Uvedomujeme si, že neexistuje jedno konkrétne riešenie, ktoré sa dá nazvať objektívne najlepším. Veríme, že dokážeme ale dospieť ku konkrétnemu záveru ktoré 
metódy (prípadne kombinácie metód) by boli najvhodnejšie pre jednotlivé prípady využitia. 

4. Porovnanie predpokladanej a skutočnej obsadenosti

Súčasťou projektu je aj sledovanie rozdielov medzi predpokladanou (napr. podľa predaja lístkov alebo historických dát) a reálnou obsadenosťou (podľa senzorických dát).
Tento aspekt umožní overiť efektívnosť vyššie rozoberaných metód zberu dát a za predpokladu presnosti umožnia dopravcom lepšie optimalizovať plánovanie spojov. 
➡️ Reálnosť dosiahnutia: Plné naplnenie tohto cieľa by si vyžadovalo prístup k reálnym dátam dopravcu. Praktické prevedenie preto v tejto fáze projektu nie je možné.
V rámci akademického prostredia je však možné realizovať príklad s vygenerovanými dátami, čo umožní demonštrovať metodiku a prípadne výsledky porovnania. 

5. Návrh spôsobu spracovania a prezentácie dát

Posledným cieľom je navrhnúť spôsob spracovania získaných dát a ich sprístupnenie dopravcom aj verejnosti. To zahŕňa návrh spôsobu odosielania, ukladania, spracovania a zobrazovania získaných 
dát uživateľom a dopravcom. 
➡️ Reálnosť dosiahnutia: Tento cieľ je reálne dosiahnuteľný na úrovni konceptuálneho návrhu. Uvedomujeme si, že konkrétna implementácia by sa pravdepodobne líšila medzi jednotlivými dopravcami. 
\caption{Rozhodujúci argument.}
\label{f:rozhod}
\end{figure*}



\section{Súčasný stav a existujúce riešenia} \label{ina}
Problematika merania obsadenosti vozidiel verejnej dopravy patrí v súčasnosti medzi vysoko aktuálne a dôležité témy v oblasti ITS (Intelligent Transport Systems) a smart mobility.
Mnohé zahraničné mestá zaviedli svoje vlastné systémy na sledovanie obsadenosti verejnej dopravy a ešte viac plánuje alebo aktívne nasadzuje svoje vlastné riešenia. V miestach so zavedenými 
riešeniami vieme zreteľne pozorovať zlepšenie optimalizácie liniek verejnej dopravy, čo priamo vedie k zvýšeniu kvality služby pre cestujúcich.

Aktuálne využívané riešenia
 - Infračervené a optické senzory - najčastejším riešením v autobusovej a železničnej doprave. Senzory počítajú vstupy a výstupy cestujúcich pri všetkých vstupných dverách vozidla,
									čím dokážu sledovať počet ľudí na palube. Praktické využitie systémov ukázalo, že sú relatívne presné (cca. 80 percent). Toto riešenie poskytuje
									systém 'Passenger Counting' od spoločnosti DILAX alebo IRIS IRMA - v súčasnosti uplatňované napríklad v mestách Berlín či Praha.
 - Kamerové systém - modernejšie riešenia využívajú umelú inteligenciu a algoritmy na spracovania obrazových záznamov v reálnom čase (OpenCV, YOLO, TesnorFlow). Kamery sú najviac uplatňované 
					vo vlakoch a metrách, kde sa vyžaduje flexibilita a možnosť spätnej verifikácie cez manuálnu kontrolu záznamov. 
 - Sieťové zariadenia - tento prístup spočíva v sledovaní počtu aktívnych mobilných zariadení pripojených na palubnú sieť Wi-Fi alebo Bluetooth. V súčastnosti sa tento systém experimentálne 
 						využíva v Švajčiarsku, či Holandsku.
 - Dáta z predajných a rezervačných systémov - predstavuje najpresnejší spôsob odhadovania obsadenosti vozidla. Tento systém však funguje iba v prípade, ak dané vozidlo má iba fixnú kapacitu
												,ktorá sa napĺňa rezerváciami, resp. kúpou lístka. Pre mestskú dopravnú sieť neposkytuje tento systém dostatočnú flexibilitu na to, aby nenarušil
												plynulý chod liniek. Napriek tomu sa jedná o najrozšírenjších spôsobov získavania dát o obsadenosti po celom svete a to najmä v železničnej doprave.
\subsection{Skúsenosti v praxi} \label{ina:nejake}
V posledných rokoch boli realizované viaceré projekty zamerané na inteligentné monitorovanie obsadenosti: 

zistit, vymenovat a opisat impleentovane systemy... (chat nasiel IDS JMK, IDS BK, Detsche Bahn, Projekt EU Horizon "MaaS4EU")

\subsection{Zhodnotenie aktualnej situacie}
Na základe aktuálneho kontextu možeme povedať, že: 
 - existuje niekolko overenych technologii, ktore je mozne adaptovat pre sledovanie obsadenosti regionalnej dopravy
 - na slovensku zacinaju iniciativy pre podobne riesenia
 - aktulane vsak chyba vztah medzi ziskavanim, spracovanim a publikovanim dat za ucelom zlepsenia kvality sluzby verejnej dopravy

Projekt nadväzuje na doterajšie poznatky v oblasti inteligentnych dopravnych systemov a poskytnut nahlad do moznosti praktickej implementacie a prispobeniu regionalnym podmienkam.??????????
Niekedy treba uviesť zoznam:

\begin{itemize}
\item jedna vec
\item druhá vec
	\begin{itemize}
	\item x
	\item y
	\end{itemize}
\end{itemize}

Ten istý zoznam, len číslovaný:

\begin{enumerate}
\item jedna vec
\item druhá vec
	\begin{enumerate}
	\item x
	\item y
	\end{enumerate}
\end{enumerate}


\subsection{Ešte nejaké vysvetlenie} \label{ina:este}

\paragraph{Veľmi dôležitá poznámka.}
Niekedy je potrebné nadpisom označiť odsek. Text pokračuje hneď za nadpisom.



\section{Dôležitá časť} \label{dolezita}




\section{Ešte dôležitejšia časť} \label{dolezitejsia}



\section{Záver} \label{zaver} % prípadne iný variant názvu



%\acknowledgement{Ak niekomu chcete poďakovať\ldots}


% týmto sa generuje zoznam literatúry z obsahu súboru literatura.bib podľa toho, na čo sa v článku odkazujete
\bibliography{literatura}
\bibliographystyle{plain} % prípadne alpha, abbrv alebo hociktorý iný
%\end{document}
